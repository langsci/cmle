%% -*- coding:utf-8 -*-


%%%%%%%%%%%%%%%%%%%%%%%%%%%%%%%%%%%%%%%%%%%%%%%%%%%%
%%%                                              %%%
%%%     Language Science Press Master File       %%%
%%%         follow the instructions below        %%%
%%%                                              %%%
%%%%%%%%%%%%%%%%%%%%%%%%%%%%%%%%%%%%%%%%%%%%%%%%%%%%

% please fill in some information in the following lines as soon
% as you have it
% Everything following a % is ignored
% Some lines start with %. Remove the % to include them

\documentclass[number=??                 %replace by your number in series
                ,series=dummyseries,     % Choose series abbreviation as appropriate
                ,isbn=xxx-x-xxxxxx-xx-x, %add your isbn here
                ,url=http://langsci-press.org/catalog/book/0,  %change to the running number of your book
	        ,output=long             % long|short|inprep              
	        %,blackandwhite
	        %,smallfont
	        ,draftmode  
		  ]{langsci}    



%%%%%%%%%%%%%%%%%%%%%%%%%%%%%%%%%%%%%%%%%%%%%%%%%%%%
%%%                                              %%%
%%%            General Setup                     %%%
%%%         no need to change this               %%%
%%%                                              %%%
%%%%%%%%%%%%%%%%%%%%%%%%%%%%%%%%%%%%%%%%%%%%%%%%%%%%

% \hypersetup{pdfdisplaydoctitle=true} % This should all go to *cls
% \usepackage{tabularx}
% \selectlanguage{USenglish} 
 

%%%%%%%%%%%%%%%%%%%%%%%%%%%%%%%%%%%%%%%%%%%%%%%%%%%%
%%%                                              %%%
%%%           Examples                           %%%
%%%                                              %%%
%%%%%%%%%%%%%%%%%%%%%%%%%%%%%%%%%%%%%%%%%%%%%%%%%%%%
% remove the percentage signs in the following lines
% if your book makes use of linguistic examples

\usepackage{lsp-gb4e} 
%% to add additional information to the right of examples, uncomment the following line
% \usepackage{jambox}
%% if you want the source line of examples to be in italics, uncomment the following line
% \def\exfont{\it}

%%%%%%%%%%%%%%%%%%%%%%%%%%%%%%%%%%%%%%%%%%%%%%%%%%%%
%%%                                              %%%
%%%          Trees                               %%%
%%%                                              %%%
%%%%%%%%%%%%%%%%%%%%%%%%%%%%%%%%%%%%%%%%%%%%%%%%%%%%

% For trees, uncomment the following lines
% \usepackage{tikz-qtree}
% % has strange side effects
% %\tikzset{every tree node/.style={align=left, anchor=north}}
% \tikzset{every roof node/.append style={inner sep=0.1pt,text height=2ex,text depth=0.3ex}}

%%%%%%%%%%%%%%%%%%%%%%%%%%%%%%%%%%%%%%%%%%%%%%%%%%%%
%%%                                              %%%
%%%      Optimality Theory                       %%%
%%%                                              %%%
%%%%%%%%%%%%%%%%%%%%%%%%%%%%%%%%%%%%%%%%%%%%%%%%%%%%
% If you are using OT, uncomment the following lines      
% % OT pointing hand
% \usepackage{pifont}
% \newcommand{\hand}{\ding{43}}
% % OT tableaux                                                
% \usepackage{pstricks,colortab}    

%%%%%%%%%%%%%%%%%%%%%%%%%%%%%%%%%%%%%%%%%%%%%%%%%%%%
%%%                                              %%%
%%%       Attribute Value Matrices               %%%
%%%                                              %%%
%%%%%%%%%%%%%%%%%%%%%%%%%%%%%%%%%%%%%%%%%%%%%%%%%%%%
%If you are using Attribute-Value-Matrices, uncomment the following lines 
% \usepackage{lsp-avm}
% \usepackage{avm}
% \avmfont{\sc} 
% \avmvalfont{\it} 
% % command to fontify the type values of an avm 
% \newcommand{\tpv}[1]{{\avmjvalfont #1}} 
% % command to fontify the type of an avm and avmspan it
% \newcommand{\tp}[1]{\avmspan{\tpv{#1}}}


%%%%%%%%%%%%%%%%%%%%%%%%%%%%%%%%%%%%%%%%%%%%%%%%%%%%
%%%                                              %%%
%%%     Discourse Representation Structures      %%%
%%%                                              %%%
%%%%%%%%%%%%%%%%%%%%%%%%%%%%%%%%%%%%%%%%%%%%%%%%%%%%
% DRS package by Alexis Dimitriadis
% \usepackage{drs}

%%%%%%%%%%%%%%%%%%%%%%%%%%%%%%%%%%%%%%%%%%%%%%%%%%%%
%%%                                              %%%
%%%            Chinese Japanese Korean           %%%
%%%                                              %%%
%%%%%%%%%%%%%%%%%%%%%%%%%%%%%%%%%%%%%%%%%%%%%%%%%%%%

% For Chinese characters, uncomment the following lines
% \usepackage[indentfirst=false]{xeCJK}
% \setCJKmainfont{SimSun}

%%%%%%%%%%%%%%%%%%%%%%%%%%%%%%%%%%%%%%%%%%%%%%%%%%%%
%%%                                              %%%
%%%               Arabic / Persian               %%%
%%%                                              %%%
%%%%%%%%%%%%%%%%%%%%%%%%%%%%%%%%%%%%%%%%%%%%%%%%%%%%

% for bidirectional text and support for Arabic/Persian, uncomment the following lines
%% \usepackage{fontspec}
% \newfontfamily\Parsifont[Script=Arabic]{XB Niloofar}
% %\usepackage{bidi}
% \usepackage{lsp-bidi}
% \newcommand{\PRL}[1]{\RL{\Parsifont #1}}
% %\TeXXeTOff
 

 
%%%%%%%%%%%%%%%%%%%%%%%%%%%%%%%%%%%%%%%%%%%%%%%%%%%%
%%%                                              %%%
%%%          additional packages                 %%%
%%%                                              %%%
%%%%%%%%%%%%%%%%%%%%%%%%%%%%%%%%%%%%%%%%%%%%%%%%%%%%

% put all additional commands you need in the 
% following files

\usepackage{localmetadata}
\usepackage{localpackages}
\usepackage{localhyphenation}
\usepackage{localcommands}

%%%%%%%%%%%%%%%%%%%%%%%%%%%%%%%%%%%%%%%%%%%%%%%%%%%%
%%%                                              %%%
%%%               END PREAMBLE                   %%%
%%%                                              %%%
%%%%%%%%%%%%%%%%%%%%%%%%%%%%%%%%%%%%%%%%%%%%%%%%%%%%
% -----------------------------------------------%%%
%%%%%%%%%%%%%%%%%%%%%%%%%%%%%%%%%%%%%%%%%%%%%%%%%%%%
%%%                                              %%%
%%%             BEGIN DOCUMENT                   %%%
%%%                                              %%%
%%%%%%%%%%%%%%%%%%%%%%%%%%%%%%%%%%%%%%%%%%%%%%%%%%%%      
\begin{document}       
%%%%%%%%%%%%%%%%%%%%%%%%%%%%%%%%%%%%%%%%%%%%%%%%%%%%
%%%                                              %%%
%%%             Frontmatter                      %%%
%%%                                              %%%
%%%%%%%%%%%%%%%%%%%%%%%%%%%%%%%%%%%%%%%%%%%%%%%%%%%%        
\maketitle                
\frontmatter
% %% uncomment if you have preface and/or acknowledgements
% \chapter*{Preface} 
% \addcontentsline{toc}{chapter}{Preface}
\chapter*{Preface}

This book is dedicated to the study of case -- an inflectional category system for marking the relations between events and the roles of their participants. However, I have to confess that it wasn't exactly love at first sight between case marking\is{case!case marking} and me. Or second. I can still recite the complete Latin\is{Latin} case paradigm\is{case!case paradigm} without batting an eyelash because me and my fellow pupils were drilled to do just that: \textit{ -us -a -um -i -ae -a}. Later, at the age of sixteen, I had an unpleasant encounter with what Mark Twain described as {\em ``that awful German\is{German} language''}. One time you had to say {\em den} and the other {\em dem} without any obvious reason. When I asked the teacher about it, I was literally told to just learn the dialogues in the book by rote and trust him that I was saying the right thing. In the meantime, English\is{English} and French\is{French} were stealing my heart because they opened new worlds to me without making a fuss about what seemed to be tiny little details at the time.

And yet, here I am presenting a book about the origins and evolution of case systems. You may interpret this as an unhealthy tendency towards masochism, but I am in fact making amends for my early prejudices against case. While working on my doctoral research, it dawned on me that case systems\is{case!case system} are very elegant solutions to a very complex communicative problem. Case marker\is{case!case marking}s turn out to be grammar's Swiss army knife\is{Swiss army knife|see{case}}: they can be used for expressing event structure\is{event structure}\is{event structure}, spatial\is{spatial language} and temporal\is{tense}\is{temporal|see{tense}} relations, gender\is{gender} and number\is{number} distinctions, and many other subtle grammatically relevant meanings. I marveled at this unexpected display of functionality and I got intrigued by the rise and fall of case paradigms\is{case!case paradigm}.

The research in this book therefore tries to be a new step in unravelling the secrets of case systems by modeling {\em how} such systems may emerge as the result of the processes whereby language users continuously (re)shape their language in locally situated communicative interactions. Since these processes are virtually impossible to grasp in natural languages, this book offers additional evidence through agent-based models in which a population of embodied artificial agents self-organize a case-like grammar with similar properties as found in case languages such as German, Latin and Turkish.

More specifically, two innovative experiments are reported. The first experiment offers the first multi-agent simulations ever that involve polysemous linguistic categories. Here, the agents are capable of inventing grammatical markers for indicating event structure relations, and of generalizing those markers to semantic roles by performing analogical reasoning over events. Extension by analogy occurs as a side-effect of the need to optimize communicative success. In the second experiment, agents are capable of combining case markers into larger argument structure constructions through pattern formation. The results show that languages become unsystematic if the linguistic inventory is unstructured and contains multiple levels of organization. This book demonstrates that this problem of systematicity can be solved through multilevel alignment. All the experiments are implemented in Fluid Construction Grammar (FCG)\is{Fluid Construction Grammar}, and use the first computational formalization of argument structure in a construction-based approach that works for both production and parsing.

Even though the experiments involve the formation of artificial languages, the results are highly relevant for natural language research as well. This book therefore engages in an interdisciplinary dialogue with linguistics and contributes to some currently ongoing debates such as the formalization of argument structure in construction grammar, the organization of the linguistic inventory, the status of semantic maps and thematic hierarchies and the mechanisms for explaining grammaticalization.


% \section*{Acknowledgements} 
% \include{chapters/acknowledgements}
\tableofcontents      
\mainmatter         

%%%%%%%%%%%%%%%%%%%%%%%%%%%%%%%%%%%%%%%%%%%%%%%%%%%%
%%%                                              %%%
%%%             Chapters                         %%%
%%%                                              %%%
%%%%%%%%%%%%%%%%%%%%%%%%%%%%%%%%%%%%%%%%%%%%%%%%%%%%

\include{chapters/introduction}  %add a percentage sign in front of the line to exclude this chapter from book
\chapter{Generativism}
\section{Chomsky}
\subsection{The early Chomsky}
\citet{Chomsky1957} can be considered the seminal\footnote{
From Latin {\em sēminālis}.
} work.
\chapter{Typology}
\section{The early days of typology}
 ...
\section{The 80s}
\subsection{Comrie}
\citet{Comrie1981} provides a good overview of fundamental concepts.

\ea                                              %numbers the example
\langinfobreak{Dutch}{personal knowledge}{}        %example metadata

\gll Dit is een hond \\                          %example source line. Do not forget the final \\
     \textsc{dem.prox} is a dog\\                %example IMT line. Do not forget the final \\
\glt `This is a dog.'                            %example translation line  
\z                                               %closes the example 
% This file will not be part of the book until you remove the initial percentage sign in lsp-skeleton.tex %uncomment to include this file in your book
\include{chapters/yetanotherfilename} 
%you can add additional chapters below if you want to 

%%%%%%%%%%%%%%%%%%%%%%%%%%%%%%%%%%%%%%%%%%%%%%%%%%%%
%%%                                              %%%
%%%             Backmatter                       %%%
%%%                                              %%%
%%%%%%%%%%%%%%%%%%%%%%%%%%%%%%%%%%%%%%%%%%%%%%%%%%%%
\backmatter
\bibliography{mybibliography.bib} %change to the name of your bib file
\end{document} 

%%%%%%%%%%%%%%%%%%%%%%%%%%%%%%%%%%%%%%%%%%%%%%%%%%%%
%%%                                              %%%
%%%                  END                         %%%
%%%                                              %%%
%%%%%%%%%%%%%%%%%%%%%%%%%%%%%%%%%%%%%%%%%%%%%%%%%%%%

% you should be able to create a pdf from this file 
% with the following command 
% xelatex lsp-skeleton.tex
% If this does not work, please get in contact with 
% Language Science Press
