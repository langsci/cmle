
\addcontentsline{toc}{chapter}{Appendix: Measures}
\chapter*{Measures}
\label{a:measures}

The simulations reported in this thesis make use of a number of measures for assessing the progress made during the experiments. This appendix collects and explains them all both in order to provide the reader with a clear understanding of what is being measured and in order to provide the research community with clear definitions of measures for future experiments.

\section*{Communicative success}

\noindent{\bf Communicative success as a local measure.} Communicative success can be measured by the agents themselves and it can influence their linguistic behaviour. In the description games played in the experiments in this thesis, a game is a success if the hearer signals agreement with the speaker's description and a failure if the hearer signals disagreement. The hearer will agree if interpretation yields a single set of bindings between the parsed meaning and the facts in the memory. The hearer will disagree if interpretation is ambiguous (i.e. more than one hypothesis was returned) or if interpretation failed.
\\
\\
\noindent{\bf Plotting communicative success.} Communicative success can also be plotted for a series of interactions by recording the success or failure of every language game. This is a global measure which is not observable by the agents themselves and thus has no influence on their linguistic behaviour. Each successful game is counted as 1 and each failed game is counted as 0. The sum of these results is then divided by the size of a certain interval into a single number between 0 and 1. The interval in all the reported simulations is set to 10.

\begin{align*}
\text{Result of game}_i &= \left\{ 
\begin{array}{ll} 
1 & \text{if game}_i \text{is successful} \\
0 & \text{if game}_i \text{is successful}
\end{array} \right.
\\
\text{Communicative success}
\begin{array}{ll}
n \\
m
\end{array}
 &= \frac{1}{(n-m)} \sum_{i=m}^n \text{Result of game}_i
\end{align*}

\section*{Cognitive effort}

\noindent{\bf Cognitive effort as a local measure.} Local cognitive effort is defined in this thesis as the number of inferences the hearer has to make during interpretation (i.e. the number of variables that need to be made equal). Since the event types in the simulations take a maximum of three participant roles, this measure ranges from 0 to 3. This number is recalculated onto a scale between 0 and 1 by taking the effort and dividing it by the maximum number of inferences (which is 3). One inference thus returns 0.33, two inferences 0.66 and three inferences 1. Failed language games count as 1, which is the maximum effort score. The agents use cognitive effort as one of the triggers for expanding their language.
\\
\\
\noindent{\bf Plotting cognitive effort.} Cognitive effort can be plotted for a series of interactions by recording the hearer's effort during each interaction. Again, this is a global measure which is only accessible for the experimenter but not for the agents themselves. As with communicative success, cognitive effort in each game returns a value between 0 and 1. Global cognitive effort is measured by dividing the sum of the results by the size of a certain interval (which here is 10). This returns a measure between 0 and 1. In the following formulae CG$_i$ stands for `the hearer's cognitive effort during game$_i$'.

\begin{align*}
\text{CG}_i &= \left\{
\begin{array}{ll}
\text{Success}_i &= \begin{array}{cc}
\text{number of inferences} \\ \hline
\text{maximum number of inferences}
\end{array}
\\ 
\text{Failure}_i &= 1
\end{array} \right.
\\ \text{Cognitive effort}
\begin{array}{ll}
n \\
m
\end{array}
 &= \frac{1}{(n-m)} \sum_{i=m}^n \text{CG}_i
\end{align*}

\section*{Average preferred lexicon}
\label{a:lexicon}

The average preferred lexicon is used by various measures in this thesis. This lexicon is derived by taking the most frequent form for every possible meaning in the population. For example, if six agents in a population of ten prefer the marker {\em -bo} for the participant role `move-1' as opposed to four agents that prefer the marker {\em -ka}, then {\em -bo} is listed in the average preferred lexicon with a frequency of 0.6. This lexicon is calculated for each individual participant role and for each possible combination of participant roles. For the experiments in this thesis, the complete meaning space of participant roles consists of the following meanings (of which the numbers correspond to the numbers in Figures \ref{f:1-coherence-1000}, \ref{f:1-coherence-7000}, \ref{f:2d-coherence-1000} and \ref{f:2d-coherence-7000}):

\begin{multicols}{2}
\begin{enumerate}
\item object-1
\item move-1
\item visible-1
\item approach-1
\item approach-2
\item distance-decreasing-1
\item distance-decreasing-2
\item fall-1
\item fall-2
\item grasp-1
\item grasp-2
\item hide-1
\item hide-2
\item move-inside-1
\item move-inside-2
\item move-outside-1
\item move-outside-1
\item touch-1
\item touch-2
\item walk-to-1
\item walk-to-2
\item cause-move-on-1
\item cause-move-on-2
\item cause-move-on-3
\item give-1
\item give-2
\item give-3
\item take-1
\item take-2
\item take-3
\item approach-1 approach-2
\item distance-decreasing-1 distance-decreasing-2
\item fall-1 fall-2
\item grasp-1 grasp-2
\item hide-1 hide-2
\item move-inside-1 move-inside-2
\item move-outside-1 move-outside-2
\item touch-1 touch-2
\item walk-to-1 walk-to-2
\item cause-move-on-1 cause-move-on-2
\item cause-move-on-1 cause-move-on-3
\item cause-move-on-2 cause-move-on-3
\item give-1 give-2
\item give-1 give-3
\item give-2 give-3
\item take-1 take-2
\item take-1 take-3
\item take-2 take-3
\item cause-move-on-1 cause-move-on-2 cause-move-on-3
\item give-1 give-2 give-3
\item take-1 take-2 take-3
\end{enumerate}
\end{multicols}

\section*{Meaning-form coherence}

Meaning-form coherence is a global measure which is not accessible to the agents. It takes the most frequent form for a particular meaning (i.e. the form which is preferred by most agents in the population) from the average preferred lexicon. For example, if the marker {\em -bo} is preferred by six agents in a population of ten agents, it is listed in the preferred average lexicon with a frequency score of 0.6. Meaning-form coherence calculates the average of all these individual frequency scores:

\begin{align*}
\text{MF coherence} &= \begin{array}{cc}
\text{sum of all frequency scores in preferred average lexicon} \\ \hline
\text{number of entries in preferred average lexicon}
\end{array}
\end{align*}

\section*{Systematicity}

Systematicity is again a global measure which is not accessible to the agents. It is calculated by taking each meaning in the average preferred lexicon and comparing it to the combinations of meanings in which it occurs. If the combination uses the same marker as the relevant meaning, then a score of 1 is counted. If it is not, a score of 0 is counted. The sum of all these scores is divided by the number of meanings that had to be checked in the average lexicon, which yields a score between 0 (no systematicity) and 1 (maximum systematicity).

For example, suppose that the meaning `appear-1' is most frequently marked by {\em -bo}, `appear-2' by {\em -ka} and the combination of the two as {\em -bo -si}. First we take `appear-1' and check whether its marker also occurs in the combination with appear-2: this is indeed the case so the form-meaning mapping is systematic in both constructions, which is counted as `1'. For appear-2, however, the pattern uses a different marker {\em -si} so no systematic relation exists across patterns, which is counted as 0. The combination itself does not occur in a larger pattern so it is not considered by the systematicity measure.
